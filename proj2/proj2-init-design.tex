\documentclass[a4paper,10pt]{article}
\usepackage[utf8]{inputenc}
\usepackage{epsfig}
\usepackage[linesnumbered,ruled,vlined]{algorithm2e}
\title{Project 2 Initial Design Document}
\author{Fang Zhang,\\
Qiushi Huang,\\
Zhengyang Song,\\
Zhaopeng Yan
}

\begin{document}

\maketitle


\section{File system calls (creat, open, read, write, close, unlink)}

\subsection{Design}

First we need to complete the handleSyscall function based on the given system call id. Based on the arguments every system call needs we declare the corresponding arguments for the new handle functions.

Also note the similarity between \emph{open} and \emph{create}, we just use an extra bool variable to indicate whether it is \emph{create} or \emph{open}.

We use a stack of integers to indicate the unused file descriptors so far, which is initialized to be 2 to 16 (since 0, 1 are reserved for special use) in \emph{load} function.

We also use an array of length 16 (since the number of file descriptors is just 16) to indicate the files opened so far, which is initialized to be 0 as standard input and 1 as standard output as required.

The main idea is as follows: for create, open we use FileSystem.open; for read, write, close we use OpenFile.read, OpenFile.write, OpenFile.close; for unlink we use FileSystem.remove.

\subsection{Variables and functions}
\begin{itemize}
\item int handleOpen(int namePtr, boolean create)
\item int handleRead(int fileDescriptor, int bufferPtr, int count)
\item int handleWrite(int fileDescriptor, int bufferPtr, int count)
\item int handleClose(int fileDescriptor)
\item int handleUnlink(int namePtr)
\end{itemize}

\begin{itemize}
\item OpenFile[] openedFiles
\item Stack$<$Integer$>$ unusedFileDesc
\end{itemize}

\subsection{Pseudo-code}

\begin{algorithm}
\DontPrintSemicolon
\KwIn{namePtr, create}
\KwOut{fd}
\If{ununsedFileDesc is empty}{
\Return ~-1
}
get file name from virtual memory using namePtr\;
\If{name is null}{\Return ~-1}
let file to be fileSystem.open parametered by name and create\;
\If{file is null}{\Return ~-1}
pop the top fd from unusedFileDesc\;
mark openedFile[id] to be file\;
\Return fd
\caption{UserProcess: handleOpen}
\label{algo:open}
\end{algorithm}

\begin{algorithm}
\DontPrintSemicolon
\KwIn{fd, bufferPtr, count}
\KwOut{res}
\If{fd < 0 or fd > 15 or count < 0 }{\Return ~-1}
\If{openedFiles[fd] is null}{\Return~-1}
initialize a buffer of size count\;
read count bytes to buffer from openedFiles[fd]\;
\If{res == -1}{\Return~-1}
write the content of buffer to bufferPtr\;
\If{number of bytes written is less than numebr of bytes read}{\Return ~-1}
\Return res
\caption{UserProcess: handleRead}
\label{algo:read}
\end{algorithm}

\begin{algorithm}
\DontPrintSemicolon
\KwIn{fd, bufferPtr, count}
\KwOut{res}
\If{fd < 0 or fd > 15 or count < 0 }{\Return ~-1}
\If{openedFiles[fd] is null}{\Return~-1}
initialize a buffer of size count\;
read count bytes to buffer from bufferPtr\;
\If{res == -1}{\Return~-1}
write the content of buffer to openedFiles[fd]\;
\If{number of bytes written is less than numebr of bytes read}{\Return ~-1}
\Return res
\caption{UserProcess: handleWrite}
\label{algo:write}
\end{algorithm}

\begin{algorithm}
\DontPrintSemicolon
\KwIn{fd}
\KwOut{0}
\If{fd < 0 or fd > 15 or count < 0 }{\Return ~-1}
\If{openedFiles[fd] is null}{\Return~-1}
close openedFiles[fd]\;
set openedFiles[fd] to be null;
push the fd to unusedFileDesc\;
\Return~ 0;
\caption{UserProcess: handleClose}
\label{algo:close}
\end{algorithm}

\begin{algorithm}
\DontPrintSemicolon
\KwIn{namePtr}
\KwOut{0 or -1}
get name from vrtual memory using namePtr\;
\If{name == null}{\Return ~-1}
remove name from file system\;
\If{remove succeeded}{\Return ~0}
\Else{\Return~-1;}
\caption{UserProcess: handleUnlink}
\label{algo:unlink}
\end{algorithm}

\subsection{Correctness constraints}
\begin{itemize}
\item No system call can raise an exception in the kernel.
\item halt() can only be invoked by 'root' process (for task 3).
\item System calls return -1 when error, else return as indicated in syscall.h
\item File descriptor 0, 1 initially is for standard input and output, but user process can close these descriptors.
\item If open() returns a OpenFile, then it can be accessed by user process.
\item Each process can open up to 16 concurrent files, each with a unique file descriptor. A file descriptor can be reused if the associated file is closed.
\end{itemize}

\subsection{Test strategy}
\begin{itemize}
\item Create a new file, then close it. Open the file, write something into it, then read it out, close it. Delete it at last. Test whether all the system calls are operating normally.
\item Try to open a not exist file, see what will happen.
\item Open up to 14 files in one process, test whether function well, test whether the 0 and 1 are for standard input and out seperately. Close them one by one, check whether the file descriptors are released timely.
\item Try to open up to 15 files(standard input and output not included), see what will happen.
\item Make sure that no exceptions are thrown during all above.
\end{itemize}


\section{Multiprogramming}

\subsection{Design}

We allocate pages for a new process and initialize the pageTable of that process when loadSection() is called. When we use virtual address, we first use the pageTable of the current process to translate it into physical address use getPPN(). Therefore, we can use it to implement read and write process.

\subsection{Variables and functions}

\begin{itemize}
\item unusedPPN: public static Stack<Integer> in userkernel, save the unused ppn and it can be allocated to every process.

\item unusedPPNLock: lock the unusedPPN.

\item UserPoccess::getPPN(int vpn, bool write): input a vpn, use it to get the ppn and return ppn. If write is true but the ppn is readonly then return -1.

\item UserProcess::readvirtualMemory():input vpn, translate it into ppn, then load the memory into data using the ppn.

\item UserProcess::writevirtualMemory():input vpn, translate it into ppn, then wirte the data into memory using the ppn.

\item Userprocess::loadsections():initialize the pagetable. First get the pagesize of the process, then map every vpn into ppn using the unusedppn in userkernel and save it in the pagetable.

\item Userprocess::unloadsections():free the ppn. Push it into unusedppn.

\item UserKernel::initialize(): add the code to initialize the unusedPPN and unusedPPNLock.
\end{itemize}
\subsection{Pseudo-code}

\begin{algorithm}
\caption{UserPoccess::getPPN(vpn, write)}
\KwIn{vpn, write}
\KwOut{ppn}
 \If{vpn < 0 || vpn >= numpages}
      {
           return -1\;
      }
get entry for pagetable according to vpn\;
use entry to get ppn\;
 \If{write and the page is readonly}
      {
           return -1\;
      }
return ppn\;
\end{algorithm}

\begin{algorithm}
\caption{UserPoccess::readvirtualMemory(vaddr,data,offset,length)}
\KwIn{$vaddr,data,offset,length$}
get memory from machine and convert $vaddr$ into vpn and pageoffset\;
ppn = getPPN(vpn,false)\;
 \If{ppn < 0}
      {
           return 0\;
      }
use ppn and pageoffset to get the address $paddr$\;
$res$ = min($length$, pagesize - pageoffset)\;
copy memory with address paddr into data with offset $offset$ and amount $res$\;
 \While{doesn't copy $length$ bits memory}
 {
    vpn++\;
    ppn = getPPN(vpn,false)\;
     \If{ppn < 0}
      {
           return $res$\;
      }
    use ppn and pageoffset = 0 to get the address $paddr$\;
    amount = min($length$ - $res$, pageSize)\;
   copy memory with address paddr into data with offset $offset + res$ and $amount$\;
   $res = res$ + amount\;
 }

return $res$\;
\end{algorithm}

\begin{algorithm}
\caption{UserPoccess::writevirtualMemory(vaddr,data,offset,length)}
\KwIn{$vaddr,data,offset,length$}
get memory from machine and convert $vaddr$ into vpn and pageoffset\;
ppn = getPPN(vpn,false)\;
 \If{ppn < 0}
      {
           return 0\;
      }
use ppn and pageoffset to get the address $paddr$\;
$res$ = min($length$, pagesize - pageoffset)\;
copy data with offset $offset$ into memory with address paddr and amount $res$\;
 \While{doesn't copy $length$ bits memory}
 {
    vpn++\;
    ppn = getPPN(vpn,false)\;
     \If{ppn < 0}
      {
           return $res$\;
      }
    use ppn and pageoffset = 0 to get the address $paddr$\;
    amount = min($length$ - $res$, pageSize)\;
    copy data with offset $offset + res$ into memory with address paddr and amount $res$\;
   $res = res$ + amount\;
 }

return $res$\;
\end{algorithm}

\begin{algorithm}
\caption{Userprocess::loadsections()}
require lock\;

 \If{numpages > the number of pages remain in memory}
      {
           release the lock and close it\;
           return false\;
      }
\For{$s=1;s \le$ number of sections ; $s++$}
{
     get the coff section\;
     \For{$i=1;s \le$ section's length ; $i++$}
     {
         vpn = section.getfirstvpn() + i\;
         get an unused ppn from userkernel\;
         initialize the pagetable[vpn] with vpn and ppn\;
         loadpage()\;
     }
}

\For{$i=1;s \le$ numpages ; $i++$}
{
     \If {pagetable[i] is not defined}
     {
          get an unused ppn from userkernel\;
          initialize the pagetable[i] with i and ppn\;
     }
}
release the lock\;

return true\;
\end{algorithm}

\begin{algorithm}
\caption{Userkernel::unloadsections()}
require lock\;

return all the ppn it used to userkernel\;

release the lock\;

\end{algorithm}

\begin{algorithm}
\caption{UserKernel::initialize()}

......\;

get the phsical page number g from the mechine.

push 0 to g into unusedPPN\;

\end{algorithm}


\subsection{Correctness constraints}

\begin{itemize}
\item when a process loads successfully, it will get enough pages to use.
\item different process will use different phsical address.
\end{itemize}


\subsection{Test strategy}

\begin{itemize}
\item Run program which use lots of memory to check whether it allocated correctly.
\item run multiple user processes to check whether the memory will overlap.
\end{itemize}

\section{System calls for process management (exec, join, exit)}

\subsection{Design}
Since only the first process can call halt(), so we need a int field to indicate the process id(the smaller the earlier). Use numCreated to denote the number of processes created so far to generate an id for the new one.

Use numRunning to denote the number of processes running in order to call terminate timely. In order to change them synchronized, we also dispatch a lock to them each.

Use a new field exitStatus to denote the exit status.

Use a userProcess parent to record the parent of it.

Use Uthread thread to denote the thread the machine actually runs.

Use exeception to record whether the process exits abnormal.

Use a map between id and Process processTable to record the child processes of this, in case join will be called. Also there is a lock associated.
\subsection{Variables and functions}
\begin{itemize}
\item int id
\item int exitStatus
\item parent
\item thread
\item exception
\item numCreated
\item numCreatedlock
\item numRunning
\item numRunninglock
\item processTable
\item processTablelock
\item int handleExit(int status)
\item int handleJoin(int processId, int statusPtr)
\item int handleExec(int filePtr, int argc, int argvPtr)
\item int handleHalt()
\item boolean execute(String name, String[] args)
\end{itemize}

\subsection{Pseudo-code}


\begin{algorithm}
\DontPrintSemicolon
\KwIn{None}
\KwOut{0 or -1}
\If{the calling process id is not 0}{\Return ~-1}
halt the machine\;
check whether the machine is halted by Lib.assert\;
\Return 0;
\caption{UserProcess: handleHalt}
\label{algo:halt}
\end{algorithm}

\begin{algorithm}
\DontPrintSemicolon
\KwIn{filePtr, argc, argvPtr}
\KwOut{child.id or -1}
get file name from virtual memory using filePtr\;
\If{name is null}{\Return ~-1}
new String array of length argc\;
\For{ i=0; i<argc; i++}{
new a buff of 4 bytes\;
load ith argument to buff from virtual memory\;
\If{load failed}{\Return ~-1}
load that from buff to args[i]\;
\If{load failed}{\Return ~-1}
}
create a new process child\;
set this process as the parent of new process\;
call execute(name, args)\;
\If{call failed}{\Return~-1}
\Return ~child.id
\caption{UserProcess: handleExec}
\label{algo:exec}
\end{algorithm}

\begin{algorithm}
\DontPrintSemicolon
\KwIn{name, args}
\KwOut{true or false}
load the name and args into this process\;
\If{load failed}{\Return false}
acquire and numRunninglock\;
increase numRunninglock\;
release the numRunninglock\;
create a new thread based on this process\;
fork the thread to run\;
\Return true;
\caption{UserProcess: execute}
\label{algo:execute}
\end{algorithm}

\begin{algorithm}
\DontPrintSemicolon
\KwIn{processID, statusPtr}

acquire processTablelock\;
get the child process using processID\;
release processTablelock\;

\If{child == null or child.parent != this}{ return -1\;}

child.thread.join()\;
child.parent = null\;

copy the exitStatus into statusPtr\;

return child.exception ? 0 : 1\;

\caption{UserProcess: handleJoin}
\label{algo:join}
\end{algorithm}

\begin{algorithm}
\DontPrintSemicolon
\KwIn{status}
\KwOut{0}
\For{all opened file f in opendedFiles}{
\If{ f is not null}{
close f}
}
release the resources taken up by the process\;
set the exitStatus to be status\;
acquire the numRunninglock\;
decrease the numRunning;
\If{numRunning is 0}{terminate the kernel}
release the numRunninglock\;
finish this process\;
check it is successfully finished by calling Lib.assert\;
\Return 0;
\caption{UserProcess: handleExit}
\label{algo:exit}
\end{algorithm}



\subsection{Correctness constraints}
\begin{itemize}
\item No system call can raise an exception in the kernel.
\item Only the root process which its id is 0 can call machine.halt().
\item A process can only join its child process.
\item every process should have an unique process id.
\item When a process exits, it should clean up any state associates with it.
\item The last process will call terminate().
\end{itemize}

\subsection{Test strategy}

\begin{itemize}
\item run a process to execute processes and join them. Test the execute() and join() and whether the last process will call terminate().
\item run a process to join a process which is not its child and check whether it will return -1.
\item run several processes and exit them normal or abnormal and check whether them clean up all the associate state.
\item run several processes and call halt() to check whether only the root process can call halt().
\end{itemize}


\section{Lottery Scheduler}

\subsection{Design}
Based on original PriorityScheduler, we extends it into LotteryScheduler, with modified pickNextThread() function by using random function to implement lottery, modified Donation() function by using collect all tickets priority.
\subsection{Variables and functions}
\begin{itemize}
\item LotteryScheduler extends PriorityScheduler;
\item LotteryQueue extends ThreadQueue;
\item ThreadState extends PriorityScheduler.ThreadState;
\item LotteryQueue.waitQueue; hashset of $<$ThreadState, Integer$>$;(use iterator to pick)
\item LotteryQueue.sum; sum of all tickets;
\item ThreadState.waitQueue; linkedlist of $<$LotteryQueue$>$;(use iterator to donate)
\end{itemize}
\subsection{Pseudo-code}
%
%\begin{algorithm}
%\caption{LotteryScheduler.increasePriority()}
%disable interrupt\;
%thread$\leftarrow$currentThread\;
%enable interrupt\;
%\end{algorithm}
%
\begin{algorithm}
\caption{LotteryQueue.pickNextThread()}
total$\leftarrow 0$\;
\If{waitQueue.size$=0$}{return null\;}
num$\leftarrow$random().nextInt(TicketsSum+1)\;
\While{num $>$ total \&\& waitQueue.hasNext()}{result$\leftarrow$waitQueue.next()\; total$+=$waitQueue.get(result).tickets\;}
return result\;
\end{algorithm}

\begin{algorithm}
\caption{ThreadState.Donation()}
disable interrupt\;
sum$\leftarrow$priority\;
queueSum$\leftarrow 0$\;
\For{any owned LotteryQueue lq}{
queueSum $=$ lq.getSum()\;
\If{Integer.MAX\_VALUE $-$ sum $<$ queueSum}{return Integer.MAX\_VALUE\;}
sum$+=$queueSum\;
}
enable interrupt\;
return sum\;
\end{algorithm}
\subsection{Correctness constraints}
\begin{itemize}
\item \#Tickets can be very large, and not exceed to MAX\_VALUE;
\item Effective priority should be sum of all donated tickets plus its own;
\end{itemize}

\subsection{Test strategy}
\begin{itemize}
\item test whether probability distribution is right;
\item test situation sum of tickets exceeding MAX\_VALUE;
\item test small number scheduler without donation;
\item test large number scheduler without donation;
\item test small number scheduler with donation;
\item test large number scheduler with donation;
\end{itemize}


\end{document}
